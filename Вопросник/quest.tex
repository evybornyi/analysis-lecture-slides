\documentclass[12pt, a4paper]{article}

\usepackage{indentfirst}%красноя строка

\usepackage[unicode,colorlinks=true, linkcolor=blue, citecolor=blue]{hyperref}%ссылки

\usepackage[utf8]{inputenc}
\usepackage[english,russian]{babel}
\usepackage{mflogo}
\usepackage{amsmath,amsfonts,amssymb}
\usepackage{xcolor}
\usepackage{graphicx}
\usepackage{cite}

\usepackage{caption}%номера рисунков
%\renewcommand{\thefigure}{\arabic{figure}}

\usepackage{amsthm,amsmath}

\renewcommand{\qedsymbol}{$\blacksquare$}
\frenchspacing
\textwidth=16.5cm
\voffset-2.5cm
\hoffset-2cm
\textheight=26cm


\renewcommand{\Im}{\mathop{\mathrm{Im}}\nolimits}
\renewcommand{\Re}{\mathop{\mathrm{Re}}\nolimits}

\newcounter{nbilet}
\newcommand{\BILET}{\par\addtocounter{nbilet}{1}%
\textbf{\bfseries БИЛЕТ №\,\arabic{nbilet}}}


\newcommand{\printbilet}[2]{
\vskip-5pt
\fbox{
\begin{minipage}[t][6cm]{\textwidth}
\vskip1em
\BILET \par
\begin{enumerate}
\item  \gettext{#1}
\item  \gettext{#2}
\end{enumerate}
\end{minipage}}
}

\usepackage{etoolbox}

\newcounter{cnt}
\newcommand\textlist{}
\newcommand\addtext[1]{%
    \stepcounter{cnt}%
    \csdef{text\thecnt}{#1}
}
\newcommand\gettext[1]{
      \csuse{text#1}
}

\setlength{\fboxsep}{10pt}

\begin{document}

%1.
\addtext{Расскажите о числах: натуральных, целых, действительных. Сформулируйте аксиому полноты действительных чисел. Докажите, что число $\sqrt{2}$ иррациональное.
}
%2.
\addtext{Расскажите о понятии множества и отображения. Что такое суперпозиция отображений? Что такое обратное отображение и при каком условии оно существует? Как связаны графики прямой и обратной функций? Приведите пример сложной и обратной функции.
}
%3.
\addtext{Дайте определения ограниченных множеств, точной верхней и нижней грани множества ($\sup$ и $\inf$) на вещественной прямой $\mathbb{R}$. Сформулируйте теорему Больцано о существовании верхней (нижней) грани всякого множества, ограниченного сверху (снизу).
}
%4.
\addtext{Что такое последовательность? Дайте определение монотонной последовательности, ограниченной сверху (снизу) последовательности, ограниченной последовательности. Приведите примеры.
}
%5.
\addtext{Дайте определение конечного предела последовательности. Приведите примеры последовательностей, имеющих и не имеющих предел. Определите бесконечный предел последовательности. 
}
%6.
\addtext{Дайте определения сходящейся последовательности. Покажите, что предел последовательности определен однозначно. Докажите, что сходящаяся последовательность ограничена.
}
%7.
\addtext{Сформулируйте арифметические свойства предела последовательности и докажите одно из них. Расскажите о неопределенностях и приведите примеры раскрытия неопределенностей.
}
%8.
\addtext{Докажите теорему о предельном переходе в неравенствах для последовательностей. Сформулируйте лемму «о двух милиционерах». Найдите предел последовательности $a_n=\frac{\sin(n)}{n}$.
}
%9.
\addtext{Сформулируйте теорему Вейерштрасса о пределе монотонной последовательности. Найдите предел последовательности  $a_n=\frac{n}{2^n}$. Что такое бесконечно малые и бесконечно большие последовательности?
}
%10.
\addtext{Определите число $e$. Докажите сходимость соответствующей последовательности.
}
%11.
\addtext{Дайте определение частичного, верхнего и нижнего пределов последовательности. Как связаны эти понятия? Сформулируйте критерий сходимости ограниченной последовательностей в терминах верхнего и нижнего предела и в терминах частичных пределов.
}
%12.
\addtext{Дайте определение фундаментальной последовательности. Покажите фундаментальность сходящейся последовательности. Сформулируйте критерий Коши сходимости последовательности.
}
%13.
\addtext{Дайте определение конечной и бесконечной точки сгущения множества. Дайте определения пределов функции
$$\lim_{x\to x_0\ (\pm \infty,\ \infty)}f(x) = a\ (\pm\infty,\infty).$$
Приведите примеры. 
}
%14.
\addtext{Сформулируйте определения левого и правого предела функции. Сформулируйте и докажите критерий существования предела функции в терминах левого и правого предела.
}
%15.
\addtext{Сформулируйте определение предела функции по Коши и по Гейне. Сформулируйте теорему об их эквивалентности. Покажите, что не существует предела $\displaystyle \lim_{x\to0} \sin\frac{1}{x}$.
}
%16.
\addtext{Сформулируйте основные свойства предела функции: однозначность, арифметические свойства, переход к пределу в неравенствах, свойство сохранения знака. Докажите одно из них.
}
%17.
\addtext{Сформулируйте и докажите теорему о замене переменных в пределе. Приведите пример.
}
%18.
\addtext{Дайте определение функции, непрерывной в точке. Что означает непрерывность функции на интервале и отрезке? Докажите, что $\sin x$ --- непрерывная функция.
}
%19.
\addtext{Расскажите об арифметических свойствах непрерывных функций. Приведите классификацию точек разрыва функции (с примерами). 
}
%20.
\addtext{Расскажите о бесконечно малых и бесконечно больших функциях, их свойствах. В каком случае функция $2^{-x}$ является бесконечно малой (бесконечно большой)? Дайте определение записи $f(x) = o(g(x))$ . Сравните асимптотическое поведение $\ln x$, $x^n$, $e^x$ при $x\to +\infty$.
}
%21.
\addtext{Дайте определение асимптотической эквивалентности. Докажите, что если $a(x)\sim b(x)$  и $c(x)\sim d(x)$ , то $a(x) c(x)\sim b(x) d(x)$ и $a(x)/c(x)\sim b(x)/d(x)$. Верно ли, что $a(x)+c(x)\sim b(x)+d(x)$? Приведите примеры.
}
%22.
\addtext{Сформулируйте и докажите теорему о первом замечательном пределе. Вычислите предел $\displaystyle\lim_{x\to0}\frac{\cos x -1}{x^2}$.
}
%23.
\addtext{Сформулируйте и докажите теорему о втором замечательном пределе. Докажите, что $e^x-1\sim x$, $\ln(1+x)\sim x$, $(1+x)^\alpha-1\sim \alpha x$  при $x\to0$.
}
%24.
\addtext{Сформулируйте и докажите теорему о промежуточном значении, дайте геометрическую интерпретацию. Изложите метод деления отрезка пополам для решения уравнения $f(x)=0$.
}
%25.
\addtext{При каком условии для функции, непрерывной на отрезке, существует непрерывная обратная функция. Сформулируйте и докажите соответствующую теорему.
}
%26.
\addtext{Сформулируйте теорему Вейерштрасса о наибольшем (наименьшем) значении непрерывной функции. Покажите на примере существенность условий теоремы.
}
%27.
\addtext{Дайте определение дифференциала и производной функции. Объясните геометрический и физический смысл производной. Как связаны понятия дифференциала и производной?
}
%28.
\addtext{Определите понятие касательной к графику функции и выведите её уравнение. Что такое односторонние производные? Приведите примеры недифференцируемых функций (с конечными односторонними производными и без). 
}
%29.
\addtext{Вычислите по определению производные следующих функций: $y=C,\ y=x^a,\ y=\sin x, y=\cos x,\ y=e^x$. Определите старшие производные функций.
}
%30.
\addtext{Расскажите, как связаны между собой понятие непрерывности и дифференцируемости функции? Приведите пример. Докажите теорему об арифметических свойствах производной.
}
%31.
\addtext{Докажите теорему о производной суперпозиции функций. Приведите примеры.
}
%32.
\addtext{Докажите теорему о производной обратной функции. Вычислите $\left( \arcsin x\right)'$, $\left( \arctg x\right)'$, $\left( \ln x\right)'$.
}
%33.
\addtext{Функция, заданная неявно или параметрически. Приведите примеры. Как находить производные в этих случаях?
}
%34.
\addtext{Дайте определение и выведите необходимое условие локального экстремума (теорема Ферма). Приведите примеры, показывающие существенность условий теоремы Ферма.
}
%35.
\addtext{Сформулируйте и докажите теорему Роля. Дайте геометрическую интерпретацию.
}
%36.
\addtext{Докажите теорему о формуле конечных приращений Лагранжа. Объясните ее геометрический смысл. Докажите, что если $f'(x)=0$ при всех $x$ из некоторого интервала, то $f$ постоянна на этом интервале.
}
%37.
\addtext{Сформулируйте и докажите теорему о формуле конечных приращений Коши. Дайте геометрическую интерпретацию. Приведите пример.
}
%38.
\addtext{Выведите необходимое и достаточное условие возрастания (убывания) функции на промежутке в терминах ее первой производной. Приведите примеры.
}
%39.
\addtext{Дайте определение многочлена Тейлора. Докажите теорему об остаточном члене в форме Пеано. Получите стандартные разложения для функций $e^x$, $\sin x$, $\cos x$, $(1+x)^a$.
}
%40.
\addtext{Докажите утверждение о том, что многочлен Тейлора дает наилучшее среди всех многочленов приближение функции в малой окрестности заданной точки. Найдите многочлены Тейлора для $\ln x$ и $\arctg x$.
}
%41.
\addtext{Напишите формулу Тейлора для функции одной переменной с остаточным членом в форме Лагранжа. Приведите альтернативное определение числа $e$ и сравните скорость сходимости соответствующих последовательностей.
}
%42.
\addtext{Выведите достаточные условия экстремума по первой производной. Расскажите о достаточном условии экстремума по старшим производным. Приведите примеры.
}
%43.
\addtext{Функция, выпуклая (вогнутая) на интервале. Выведите условия выпуклости (вогнутости) функции с точки зрения первой и второй производных. Приведите примеры. Дайте геометрическую интерпретацию выпуклости с точки зрения расположения хорд и касательных.
}
%44.
\addtext{Дайте определения вертикальной и наклонной асимптот функции. Выведите условия существования наклонной асимптоты. Выведите формулы для её нахождения. Приведите пример.
}
%45.
\addtext{Сформулируйте правило Лопиталя, докажите его в случае неопределенности вида $\frac{0}{0}$. Приведите примеры.
}

%---------------------------------------------------------------------------------------------------
%46.
\addtext{Дайте определение неопределенного интеграла (первообразной) и укажите его основные свойства. Докажите утверждение об общем виде первообразной заданной функции. Приведите пример.
}
%47.
\addtext{Опишите с обоснованием методы замены переменной и подведения под знак дифференциала в неопределённом интеграле. Найдите первообразную для $\sqrt{1+x^2}$.
}
%48.
\addtext{Расскажите с обоснованием об интегрировании по частям в неопределённом интеграле. Найдите первообразную функции $\ln(x)$.
}
%49.
\addtext{
Сформулируйте теорему о представлении рациональной функции в виде суммы простых дробей. Расскажите об интегрировании рациональной функции. Приведите пример.}
%50.
\addtext{
Расскажите, как следующие интегралы сводятся к интегралам от рациональных функций:
$$\int R(x^\alpha, x^\beta,\ldots,x^\omega)dx,\qquad \int R(e^{\alpha x}, e^{\beta x},\ldots,e^{\omega x})dx,$$
где $R$ --- рациональная функция своих аргументов, а $\alpha,\ldots,\omega$  – рациональные числа. Вычислите 
$$\int \frac{e^x+1}{e^{2x}+1}dx.$$
}
%51
\addtext{Расскажите о вычислении интегралов от $R\left(x,\sqrt{\frac{ax+b}{cx+d}}\right)$, где $R$ --- рациональная функция. Приведите пример.
}
%52
\addtext{Расскажите о том, как тригонометрические интегралы вида:
$$\int R(\sin(x),\cos(x))dx,$$
где $R$ --- рациональная функция, можно привести к интегралам от рациональных функций. Приведите пример.
 }
%53
\addtext{Расскажите о вычислении интегралов от $R(x,\sqrt{ax^2+bx+c})$ при помощи подстановок Эйлера ($R$ --- рациональная функция). Приведите пример.
}
%54
\addtext{Дайте определение функции, интегрируемой на отрезке, и ее определенного интеграла. Поясните геометрический смысл определенного интеграла. Выведите свойство линейности и свойство аддитивности определенного интеграла.
}
%55
\addtext{Сформулируйте и докажите теоремы об интегрировании неравенств и об оценке модуля определённого интеграла.
}
%56
\addtext{Сформулируйте и докажите теорему о среднем значении для определенного интеграла.  В чём геометрический смысл этой теоремы? Приведите формулировку общей теоремы о среднем. 
}
%57
\addtext{Докажите теорему о производной интеграла с переменным верхним пределом и выведите формулу Ньютона–Лейбница.
}
%58
\addtext{Сформулируйте правила замены переменной и интегрирования по частям в определенном интеграле. Приведите примеры.
}
%59
\addtext{Выведете формулу Тейлора с остаточным членом в виде интеграла.
}
%60
\addtext{Приведите формулы для площади фигуры на плоскости и объёма тела (в том числе тела вращения) в пространстве. Приведите формулы для длины гладкой кривой, заданной параметрически, на плоскости и в пространстве. Как вычислить длину дуги графика функции?
}
%61
\addtext{Дайте определение несобственного интеграла 1-го рода (по бесконечному промежутку). Приведите пример сходящегося и расходящегося интеграла.
}
%62
\addtext{Дайте определение несобственного интеграла 2-го рода (от неограниченной функции по конечному промежутку).  Приведите пример сходящегося и расходящегося интеграла.
}
%63
\addtext{Расскажите с обоснованием о поведении несобственных интегралов:
$$\int_1^{+\infty} \frac{dx}{x^\alpha},\qquad \int_0^1\frac{dx}{x^\beta},\qquad \int_a^b\frac{dx}{(x-a)^\gamma}.$$
}
%64
\addtext{Сформулируйте определения абсолютной и условной сходимости несобственных интегралов. Докажите, что из абсолютной сходимости следует сходимость интеграла.
}
%65
\addtext{Сформулируйте теоремы о сравнении (асимптотическом сравнении) несобственных интегралов от положительных функций.
}
%66
\addtext{Дайте определение частичной суммы числового ряда, сходящегося числового ряда и его суммы. Сформулируйте основные свойства числовых рядов. Покажите, что если ряд сходится, то его члены стремятся к $0$. Приведите пример, показывающий, что обратное не верно.
}
%67
\addtext{Дайте определение знакопостоянного ряда. Сформулируйте признаки сравнения и асимптотического сравнения для знакопостоянных рядов.}
%68
\addtext{Сформулируйте признаки сходимости Даламбера и Коши в простой и передельной форме. Докажите один из этих признаков.
}
%69
\addtext{Выведите интегральный признак сходимости числового ряда. Исследуйте сходимость ряда Дирихле $\displaystyle \sum_{k=1}^{+\infty}\frac{1}{k^\alpha}$.
}
%70
\addtext{Дайте определение абсолютно сходящегося числового ряда. Докажите, что из абсолютной сходимости ряда следует сходимость. Приведите пример сходящегося, но не абсолютно сходящегося ряда.
}
%71
\addtext{Докажите признак Лейбница о знакочередующихся рядах. Для рядов Лейбница выведите оценку уклонения частичной суммы от суммы ряда.
}
%72
\addtext{Дайте определение поточечной и равномерной сходимости функциональной последовательности (функционального ряда). Приведите пример, показывающий, что из поточечной сходимости не следует равномерная сходимость.
}
%73
\addtext{Дайте определение степенного ряда. Сформулируйте теоремы о множестве сходимости степенного ряда и его равномерной сходимости. Приведите примеры всюду сходящегося и всюду расходящегося степенного ряда.
}
%74
\addtext{Приведите формулу Коши--Адамара для радиуса сходимости степенного ряда. Приведите пример.
}
%75
\addtext{Покажите, что внутри интервала сходимости степенной ряд можно почленно интегрировать и дифференцировать. Найдите сумму ряда
$$x+2x^2+3x^3+\cdots$$
}
%76
\addtext{Что такое ряд Тейлора и как он связан с формулой Тейлора? Докажите единственность разложения функции в сходящийся степенной ряд. Покажите, что не всякий сходящийся ряд Тейлора сходится к функции, по которой он был построен. 
}
%77
\addtext{Выведите стандартные разложения Маклорена и найдите интервалы сходимости для функций  $e^x, \sin(x), \cos(x)$.
}
%78
\addtext{Выведите стандартные разложения Маклорена и найдите интервалы сходимости для функций  $\ln(1+x), \arctg(x), (1+x)^\alpha (x)$.
}
%79
\addtext{Что такое расстояние в  $\mathbb{R}^n$? Что такое шар в  $\mathbb{R}^n$? Дайте определение предела последовательности точек в $\mathbb{R}^n$. Докажите, что сходимость последовательности точек в $\mathbb{R}^n$ эквивалентна покоординатной сходимости.
}
%80
\addtext{Дайте определения ограниченного множества, открытого и замкнутого множества в $\mathbb{R}^n$. Приведите их основные свойства. Определите границы множества, связное множество, область.
}
%81
\addtext{Расскажите о понятии функции нескольких переменных. Что такое график функции? Что такое линии и поверхности уровня функции? Дайте определение предела функции и непрерывности.
}
%82
\addtext{Дайте определение компакта, функции, непрерывной на множестве. Сформулируйте теорему Вейерштрасса о функции, непрерывной на компакте.
}
%83
\addtext{Определите частные производные первого порядка для функций многих переменных и расскажите об их арифметических свойствах. Поясните геометрический смысл частных производных в случае функции двух переменных.
}
%84
\addtext{Определите дифференциал функции многих переменных. Определите понятие касательной плоскости к графику функции двух переменных и приведите её уравнение. Сформулируйте условие на частные производные необходимое (достаточное) для дифференцируемости функции.
}
%85
\addtext{Определите понятие производной по направлению и градиента для функции многих переменных. Выведите формулу для производной по направлению. В чём геометрический смысл градиента?
}
%86
\addtext{Расскажите о частных производных старших порядков, сформулируйте теорему Шварца и приведите пример. Расскажите о правилах дифференцирования сложной функции, приведите примеры.
}
%87
\addtext{Приведите формулу Тейлора для функции многих переменных. Дайте определение старших дифференциалов функции нескольких переменных. Найдите второй дифференциал функции двух переменных.
}
%88
\addtext{Дайте определение точки локального экстремума функции нескольких переменных. Выведите необходимое условие локального экстремума для дифференцируемых функций.
}
%89
\addtext{Сформулируйте достаточные условия экстремума функции многих переменных в терминах второго дифференциала. Сформулируйте критерий Сильвестра. Приведите пример.
}
%90
\addtext{Сформулируйте теоремы о неявной функции многих переменных. Выведете формулу для частных производных функции, заданной неявно.
}
%90
\addtext{Расскажите о заменах координат в многомерном пространстве.  Дайте определение матрицы Якоби и поясните ее геометрический смысл, записав формулу локальной линеаризации отображения. Приведите пример.
}
%91
\addtext{
Сформулируйте теорему об обратимости регулярной замены координат. Выведете формулу для матрицы Якоби обратного отображения. Рассмотрите переход к полярным и сферическим координатам.
}
%92
\addtext{Дайте определение измеримого (по Жордану) ограниченного множества. Дайте определение двойного и тройного интеграла. Укажите его основные свойства (линейность, аддитивность, интегрирование неравенств) и поясните его геометрический смысл.
}
%93
\addtext{Расскажите о сведении двойного интеграла к повторному. Как вычислить интеграл от функции с мультипликативной структурой по прямоугольнику? Приведите примеры.
}
%94
\addtext{Определите якобиан отображения  плоской области  и поясните его геометрический смысл. Запишите формулу замены переменной в двойном интеграле. Рассмотрите пример.
}

\section*{Вопросы к экзамену. Часть 1}
\begin{enumerate}
\item \gettext{1}
\item \gettext{2}
\item \gettext{3}
\item \gettext{4}
\item \gettext{5}
\item \gettext{6}
\item \gettext{7}
\item \gettext{8}
\item \gettext{9}
\item \gettext{10}
\item \gettext{11}
\item \gettext{12}
\item \gettext{13}
\item \gettext{14}
\item \gettext{15}
\item \gettext{16}
\item \gettext{17}
\item \gettext{18}
\item \gettext{19}
\item \gettext{20}
\item \gettext{21}
\item \gettext{22}
\item \gettext{23}
\item \gettext{24}
\item \gettext{25}
\item \gettext{26}
\item \gettext{27}
\item \gettext{28}
\item \gettext{29}
\item \gettext{30}
\item \gettext{31}
\item \gettext{32}
\item \gettext{33}
\item \gettext{34}
\item \gettext{35}
\item \gettext{36}
\item \gettext{37}
\item \gettext{38}
\item \gettext{39}
\item \gettext{40}
\item \gettext{41}
\item \gettext{42}
\item \gettext{43}
\item \gettext{44}
\item \gettext{45}
\end{enumerate}

\section*{Вопросы к экзамену. Часть 2}
\begin{enumerate}
\setcounter{enumi}{45}
\item \gettext{46}
\item \gettext{47}
\item \gettext{48}
\item \gettext{49}
\item \gettext{50}
\item \gettext{51}
\item \gettext{52}
\item \gettext{53}
\item \gettext{54}
\item \gettext{55}
\item \gettext{56}
\item \gettext{57}
\item \gettext{58}
\item \gettext{59}
\item \gettext{60}
\item \gettext{61}
\item \gettext{62}
\item \gettext{63}
\item \gettext{64}
\item \gettext{65}
\item \gettext{66}
\item \gettext{67}
\item \gettext{68}
\item \gettext{69}
\item \gettext{70}
\item \gettext{71}
\item \gettext{72}
\item \gettext{73}
\item \gettext{74}
\item \gettext{75}
\item \gettext{76}
\item \gettext{77}
\item \gettext{78}
\item \gettext{79}
\item \gettext{80}
\item \gettext{81}
\item \gettext{82}
\item \gettext{83}
\item \gettext{84}
\item \gettext{85}
\item \gettext{86}
\item \gettext{87}
\item \gettext{88}
\item \gettext{89}
\item \gettext{90}
\item \gettext{91}
\item \gettext{92}
\item \gettext{93}
\item \gettext{94}
\end{enumerate}


\end{document}

